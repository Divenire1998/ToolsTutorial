%导言区
\documentclass{ctexart}%ctexbook,ctexrep

%\usepackage{ctex}
\usepackage{amsmath}
\usepackage{mathdots}

\newcommand{\adots}{\mathinner{\mkern2mu%
		\raisebox{0.1em}{.}\mkern2mu\raisebox{0.4em}{.}%
		\mkern2mu\raisebox{0.7em}{.}\mkern1mu}}

%正文区(文稿区)
%矩阵环境,用&分隔列,用\\分隔行
\begin{document}
	\[
	\begin{matrix}
		0 & 1 \\
		1 & 0
	\end{matrix}
	% pmatrix环境
	\begin{pmatrix}
		0 & -i\\
		i & 0
	\end{pmatrix} \qquad
	% bmatrix环境
	\begin{bmatrix}
		0 &-1 \\
		1 & 0
	\end{bmatrix} \qquad
	% Bmatrix环境
	\begin{Bmatrix}
		1 &0\\
		0 &-1
	\end{Bmatrix} \qquad
	% vmatrix环境
	\begin{vmatrix}
		a&b\\
		c&d
	\end{vmatrix} \qquad
	% Vmatrix环境
	\begin{Vmatrix}
		i &0\\
		0 &-i
	\end{Vmatrix}
	\]
	
	%可以使用上下标.
	\[
	A =\begin{pmatrix}
		a_ {11}^2 & a_ {12}^2 & a_ {13}^2 \\
		0 & a_{22} & a_{23}\\
		0 & 0 & a_{33}
	\end{pmatrix}
	\]
	
	%常用省略号:\dots、\vdots、\ddots、\adots
	\[
	A = \begin{bmatrix}
		a_{11} & \dots & a_{1n}\\
		\adots & \ddots & \vdots \\
		0 & & a_{nn}
	\end{bmatrix}_ {n \times n}
	\]
	
	%此处使用\iddots同样拥有效果,但需要引入mathdots宏包
	\[
	A = \begin{bmatrix}
		a_{11} & \dots & a_{1n}\\
		\iddots & \ddots & \vdots \\
		0 & & a_{nn}
	\end{bmatrix}_ {n \times n}
	\]
	
	%分块矩阵(矩阵嵌套)
	\[
	\begin{pmatrix}
		\begin{matrix} 1&0\\0&1 \end{matrix}
		& \text{\Large 0} \\
		\text{\Large 0} & \begin{matrix}
			1&0\\0&-1 \end{matrix}
	\end{pmatrix}
	\]
	
	%三角矩阵
	%multicolumn合并多列(2是合并数,c是位置在中间);taisebox调整高度
	\[ \begin{pmatrix}
		a_ {11} & a_ {12} & \cdots & a_{1n} \\
		& a_{22} & \cdots &a_{2n} \\
		& & \ddots & \vdots \\
		\multicolumn{2}{c}{\raisebox{1.3ex}[0pt]{\Huge 0}}
		& &a_{nn}
	\end{pmatrix}
	\]
	
	
	%跨列的省略号: \hdotsfor{<列数>}
	\[
	\begin{pmatrix}
		1&\frac12&\dots&\frac1n\\
		\hdotsfor{4} \\
		m &\frac m2 & \dots & \frac mn
	\end{pmatrix}
	\]
	
	%行内小矩阵(smallmatrix) 环境
	复数$z = (x,y)$也可用矩阵
	\begin{math}
		\left(%需要手动加上左括号
		\begin{smallmatrix}
			x&-y\\y&x
		\end{smallmatrix}
		\right)%需要手动加上右括号
	\end{math}来表示。|
	%\left(形成左括号,\right)形成右括号。
	
	% array环境(类似于表格环境tabular)
	\[
	\begin{array}{r|r}
		\frac{1}{2} & 0 \\
		\hline
		0 & -\frac a{bc} \\
	\end{array}
	\]
	
	%用array环境构造复杂矩阵
	\[
	% @{<内容>}- _添加任意内容,不占表项计数
	%此处添加一个负值空白,表示向左移- 5pt的距离
	\begin{array}{c@{\hspace{-5pt}}l}
		%第1行,第1列
		\left(
		\begin{array}{ccc|ccc}
			a&\cdots&a&b&\cdots&b\\
			& \ddots & \vdots & \vdots & \adots\\
			& & a &b\\ \hline
			& & & c & \cdots & c\\
			& & & \vdots & & \vdots\\
			\multicolumn{3}{c|}{\raisebox{2ex}[0pt]{\Huge0}}
			& c & \cdots & c
		\end{array}
		\right)
		&
		%第1行第2列
		\begin{array}{l}
			%\left.仅表示与\right\}配对,什么都不输出
			\left. \rule{ 0mm}{7mm}\right\}p\\
			\\
			\left. \rule{0mm}{7mm}\right\}q
		\end{array}
		\\[-5pt]
		%第2行第1列
		\begin{array}{cc}
			\underbrace{\rule{17mm}{0mm}}_ m &
			\underbrace{\rule{17mm}{0mm}}_ m
		\end{array}
		& %第2行第2列
	\end{array}
	\]
	
\end{document}