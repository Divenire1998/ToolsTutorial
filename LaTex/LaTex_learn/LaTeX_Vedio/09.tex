%导言区
\documentclass{ctexart}%ctexbook,ctexrep

%\usepackage{ctex}
\usepackage{graphicx}
\graphicspath{{figures/}} %图片在当前目录下的figures目录
%标题控制(caption、bicaption等宏包)
%并排与子图表(subcaption、 subfig、 floatrow等宏刨)
%绕排(picinpar、 wrapfig等宏包)


%=======浮动体============
%实现灵活分页(避免无法分割的内容产生的页面留白)
%给图表添加标题
%交叉引用

%=======figure环境(table环境与之类似)============
%\begin{figure}[<允许位置>]
%<任意内容>
%\end{figure}
% <允许位置>参数(默认tbp)
% h,此处(here)-代码所在的上下文位置
% t,页顶(top)-代码所在页面或之后页面的顶部
% b,页底(Buttom)-代码所在页面或之后页面的底部
% p,独立一页(page)-浮动页面

%正义区(文稿区)
\begin{document}
	%利用\ref进行标签的引用,进而达到交叉引用。
	\LaTeX{}中\TeX 系统的吉祥物---小狮子见图\ref{fig-lion}。
	
	%指定参数指定浮动体的排版位置
	\begin{figure} [htbp]
		\centering
		\includegraphics [scale=0.1]{lion}
		%利用fig设置标签及编号
		\caption{\TeX 系统的吉祥物---小狮子}\label{fig-lion} 
	\end{figure}
	
	遥望太白,看积雪皑皑,别有一番风景(图\ref{fig-mountain})。.
	\begin{figure}[htbp]%允许各个位置
		\centering
		\includegraphics [scale=0.1]{mountain}
		%利用fig设置标签及编号
		\caption{太白山}\label{fig-mountain}
	\end{figure}
	
	
	在\LaTeX{}中的表格:
	当然,在\LaTeX{}中也如以使用表\ref {tab-score}所示的表格。%此处对表格进行了引用。
	\begin{table}[h]
		\centering
		\caption{考试成绩单} \label{tab-score}
		
		\begin{tabular}{|l|c|c|c|r|}
			\hline
			姓名 & 语文 & 数学 & 外语 & 备注 \\
			\hline
			张三 & 87 & 100 & 93 & 优秀 \\
			\hline
			李四 & 75 & 64 & 52 & 补者另行通知 \\
			\hline
			王二 & 80 & 82 & 78 & \\
			\hline
		\end{tabular}
	\end{table}
\end{document}

