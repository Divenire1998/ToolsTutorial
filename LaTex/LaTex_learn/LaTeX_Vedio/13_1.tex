%导言区
\documentclass{article}
\usepackage{ctex}
%正文区(文稿区)
\begin{document}
	%一次管理,一次使用
	%参考文献格式:
	%\begin{thebibliography}{编号样本}
	%\bibitem[记号]{引用标志}文献条目1 (利用bibitem排版一个参考文献条目)
	%\bibitem[记号]{引用标志}文献条目2
	%...
	%\end{thebibliography}
	%其中文献条日包括:作者,题目,出版社,年代,版本,页码等。
	%引用时候要可以采用: \cite{引用标志1, 引用标志2,.}
	引用一篇文章\textsuperscript{\cite{articlel}}
	
	引用一本书\cite{book1} 引用一本书\cite{latexGuide}等等
	
	\begin{thebibliography}{99}
		\bibitem{articlel}陈立程,苏伟,蔡川,陈晓云. \emph{基于LaTex的Web数学公式提取方法研究}[J].让算机科学,2014(06)
		%利用emph命令强调参考文献中的某些内容。
		\bibitem{book1}William H. Press,Saul A. Teukolsky,
		William T. Vetterling, Brian P. Flannery,
		\emph{Numerical Recipes 3rd Edition:
			The Art of Scientific Computing}
		Cambridge University Press, New York , 2007.
		\bibitem{latexGuide} Kopka Helmut, W. Daly Patrick,
		\emph{Guide to \LaTeX}, $4^{th}$ Edition.
		Available at \texttt{http://www. amazon. com}.
		\bibitem{1atexMath} Graetzer George, \emph{Math Into \LaTeX},
		BirkhAluser Boston; 3 edition (June 22, 2000).
	\end{thebibliography}
\end{document}