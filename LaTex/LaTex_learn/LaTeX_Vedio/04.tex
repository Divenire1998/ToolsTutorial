%导言区
\documentclass[10pt]{article}
%设置基础字体大小【10磅】

\usepackage{ctex}
%格式与内容分离的字体
\newcommand{\myfont}{\textbf{\textsf{Fancy Text}}}
%正文区(文稿区)
\begin{document}
	%字体族设置(罗马字体、无衬线字体、打字机字体)

	\textrm{Roman Family} \textsf{Sans Serif Family} \texttt{Typewriter Family}
	
	%rmfamily表示后续字体均为“罗马”字体,直到遇到
	\rmfamily Roman Family {\sffamily Sans Serif Family} {\ttfamily Typewriter Family}
	
	\sffamily The love world is big, which can hold hundreds of disappointments; the love world is small which is crowded even with three people inside.
	
	\ttfamily The love world is big, which can hold hundreds of disappointments; the love world is small which is crowded even with three people inside.
	
	%字体系列设置(粗细、宽度)
	\textmd{Medium Series} \textbf{Boldface Series}
	{\mdseries Medium Series} {\bfseries Boldface Series}
	
	%字体形状(直立、斜体、伪斜体、小型大写)
	%字体设置命令
	\textup{Upright Shape} \textit{Italic Shape}
	\textsl{Slanted Shape} \textsc{Small Caps Shape}
	
	%字体设置声明
	{\upshape Upright Shape} {\itshape Italic Shape} {\slshape Slanted Shape} {\scshape Small Caps Shape}
	
	%中文字体
	{\songti 宋体} \quad{\heiti 黑体} \quad {\fangsong 仿宋} \quad {\kaishu 楷书}
	中文字体的\textbf{粗体}与\textit{斜体}。
	
	%字体大小
	{\tiny Hello}\\
	{\scriptsize Hello}\\
	{\footnotesize Hello}\\
	{\small Hello}\\
	{\normalsize Hello}\\
	{\large Hello}\\
	{\Large Hello}\\
	{\LARGE Hello}\\
	{\huge Hello}\\
	{\Huge Hello}\\
	
	%中文字号设置命令
	\zihao{-0} 你好
	
	%具体可看文档
	\myfont
\end{document}