%导言区
\documentclass{ctexbook}%ctexart,ctexbook
%\usepackage{ctex}

% ======设置标题的格式为左对齐======
% ======具体参数见ctex文档======
\ctexset{
	section = {
		format+= \zihao{-4} \heiti \raggedright,
		name = {,、},
		number = \chinese{section},
		beforeskip = 1.0ex plus 0.2ex minus .2ex,
		afterskip = 1.0ex plus 0.2ex minus .2ex,
		aftername =\hspace{0pt}
	},
	subsection = {
		format+= \zihao{5} \heiti \raggedright,
		name = {,、},
		number = \arabic{subsection},
		beforeskip = 1.0ex plus 0.2ex minus .2ex,
		afterskip = 1.0ex plus 0.2ex minus .2ex,
		aftername = \hspace{0pt}
	},
}

%正文区(文稿区)
\begin{document}
	%tableofcontents就是整个的目录
	\tableofcontents
	%加上chapter需要将ctexart改为ctexbook
	\chapter{摘要}
	我是摘要
	\section{引言}
	某君昆仲,今隐其名,皆余昔日在中学时良友;分隔多年,消息渐阙。日前偶闻其一大病;适归故乡,迂道往访,则仅晤一人,言病者其弟也。
	某君昆仲,今隐其名,皆余昔日在中学时良友;分隔多年,消息渐阙。日前偶闻其一大病;适归故乡,迂道往访,则仅晤一人,言病者其弟也。
	\section{试验方法}
	\section{实验结果}
	\subsection{数据}
	\section{结果}
	\subsection{图表}
	\subsubsection{实验条件}
	\subsubsection{实验过程}
	\subsection{结果分析}
	\section{结论}
	\section{致谢}
\end{document}