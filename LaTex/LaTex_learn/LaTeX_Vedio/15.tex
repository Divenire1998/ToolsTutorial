%导言区
\documentclass{ctexart}%ctexbook,ctexrep
% \newcommand-定 义命令
%命令只能由字母组成,不能以\end开头
% \newcommand<命 令>[<参数个数>][<首参数默认值>]{<具体定义>}

% \newcommand可以是简单字符串替换,例如:
%使用\PRC 相当于People's Republic of \emph{China} 这一串内容
\newcommand\PRC{People's Republic of \emph{China}}

% \newcommand也可以使用参数
%参数个数可以从1到9,使用时用#1,#2......#9 表示
\newcommand\loves [2]{#1喜欢#2}
\newcommand\hatedby[2]{#2不受#1喜欢}

% \newcommand的参数也可以有默认值
%指定参数个数的同时指定了首个参数的默认值,那么这个命令的
%第一个参数就成为可选的参数(要使用中括号指定)
\newcommand\love[3][喜欢]{#2#1#3}

%\renewcommand -重定义命令
%与\newcommand命令作用和用法相同,但只能用于已有命令
% \renewcommand<命令>[<参数个数>][<首参数默认值>]{<具体定义>}
\renewcommand\abstractname{内容简介}

%定义和重定义环境
%\newenvironment{<环境名称>}[<参数个数>][<首参数默认值>]
% {<环境前定义>}
% {<环境后定义>}
%\renewenvironment{<环境名称>}[<参数个数>][<首参数默认值>]
% {<环境前定义>},
% {<环境后定义>}

%为book类中定义摘要(abstract) 环境
\newenvironment{myabstract}[1][摘要]%
{\small
	\begin{center}\bfseries #1 \end{center}%
	\begin{quotation}}%
	{\end{quotation}}

%环境参数只有<环境前定义>中可以使用参数,
% <环境后定义>中不能再使用环境参数。
%如果需要,可以先把前面得到的参数保存在一个命令中,在后面使用:
\newenvironment{Quotation}[1]%
{\newcommand\quotesource{#1}%
	\begin{quotation}}%
	{\par\hfill---《\textit{\quotesource}》%
\end{quotation}}

%正文区(文稿区)
\begin{document}
	\begin{myabstract}[我的摘要]
		%此处增加默认值之后【我的摘要】会取代原参数默认值——【摘要】
		这是一段自定义格式的摘要...
	\end{myabstract}
	
	
	\begin{abstract}
		这是一段摘要...
	\end{abstract}
	
	\begin{Quotation}{易$\cdot$乾}
		初九,潜龙勿用。
	\end{Quotation}
	
	
	
	\PRC
	
	\loves{猫儿}{鱼}
	
	\hatedby{猫儿}{萝卜}
	
	\love{猫儿}{鱼}
	%猫儿是#2,鱼是#3,#1是默认参数喜欢
	
	\love[最爱]{猫儿}{鱼}
	%方括号“最爱”是#1参数。猫儿和鱼分别是#2与#3
\end{document}

%定义命令和环境是进行\LaTeX{}格式定制、达成内容与格式分离目标的利器。使用自定义的命令和环境把字体、字号、缩进、对齐、间距等各种琐细的内容包装起来,赋以一个有意义的名字,可以使文挡结构清晰、代码整洁、易于维护。在使用宏定义的功能时,要综合利用各种已有的命令、环境、变量等功能,事实上,前面所介绍的长度变量与盒子、字体字号等内容,大多并不直接出现在文档正文中,而主要都是用在实现各种结构化的宏定义里。